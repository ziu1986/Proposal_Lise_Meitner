\textbf{The project \gls{odina} will improve our understanding of the two-way feedback between ozone and vegetation in the light of climate change}. The project targets key issues in international policies and efforts directed towards air quality and climate, but also, human health and food security.
The intended developments will \textbf{bridge the gap between state-of-the-art understanding of plant physiology and climate modeling}. 
Furthermore, an assessment of the contribution of plant health on the emission of \glspl{bvoc} may \textbf{pave the way for a wide range of new insights regarding aerosol formation} and related uncertainties under future climate conditions. This could provoke investigations into improving the \gls{bvoc} emission models, tying them more tightly to the physiological state of the vegetation in \glspl{esm}.

Although ozone-induced reduction of stomatal conductance and photosynthesis has been addressed previously (as reviewed in Section~\ref{sec:review}), ozone damage on vegetation has not been included into the nutrient cost driven photosynthesis on the process level and at the same time coupled to both atmosphere and atmospheric chemistry, in \gls{cesm}.
Such \textbf{extension of the \gls{cesm}} is a timely effort given the importance of the process and the extensive model user base. The \gls{cesm} user community extends worldwide, with about $4,000$ registered users in the support forum, and its importance for scientific discovery is reflected in about $3,000$ research and development papers published since 2009.

Without doubt, \textbf{new developments in such a widely used model will have a lasting impact on the climate and air quality research communities}. The proposed research will advance our modeling capacities and improve process representation beyond what is reflected in \gls{cmip}~5 or \gls{cmip}~6 generation models, and thus contribute to a \textbf{more complete understanding of surface air quality and more robust future projections}.

