Although \textbf{\color{red}ozone damage} has been addressed previously as reviewed in Section~\ref{sec:review}, ozone damage has not been included on the process level into a nutrient cost driven photosynthesis and at the same time coupled to both atmosphere and atmospheric chemistry, in \gls{cesm}.

Such extension of the \gls{cesm} model is a timely effort given the importance of the process and the extensive model user base. The \gls{cesm} user community extends worldwide, with about $4,000$ registered users in the support forum, and its importance for scientific discovery is reflected in about $3,000$ research and development papers published since 2009.

Without doubt, new developments in such a widely used model will have a lasting impact on the climate and air quality research communities. Besides the advancement of our modeling capacities the proposed research will contribute to a more complete understanding of surface air quality, and more robust future projections, and represent an improvement beyond the capabilities reflected in \gls{cmip}~5 or \gls{cmip}~6 generation models.
