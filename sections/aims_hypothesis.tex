\glsreset{cesm}

The project \textbf{\gls{odina}} focuses on the process-based modeling of ozone induced damage on vegetation. To this end the applicant will implement a two-way coupling of ozone between the atmosphere and land-biosphere in the widely used global climate model \gls{cesm}. \gls{odina} studies the vegetations' feedback on near-surface to tropospheric ozone concentrations and will improve our process-level understanding of dry deposition effects on air quality and plant health. The improved capabilities of \gls{cesm} including the \gls{odina} model will be demonstrated and effects of air quality regulations, and biosphere feedbacks on surface ozone air quality will be studied under the high emission \gls{ssp}~5. \gls{odina} will also enable the research team to study ozone impacts through vegetation on the global carbon cycle and water cycle, and thus ultimately climate.
The \gls{odina} project is guided by \textbf{five working hypotheses}: 

\begin{enumerate}
\itemsep0pt
\item Elevated ozone concentration levels affect plant physiology via reduced stomatal conductance and photosynthesis. 
\item Reduced stomatal conductance will decrease ozone uptake by vegetation and hence diminish the dry deposition sink of ozone to the land biosphere. This will in turn act as positive feedback and lead to higher ground-level ozone concentrations in regions with damaged vegetation, and thus degrade ambient air quality.
\item Decreasing stomatal conductance will further affect transpiration of plants implicating lower relative humidity and less cooling by latent heat. At the same time more water will remain in the soil.
\item A reduction of photosynthesis will affect the carbon sequestration and the terrestrial carbon sink, and hence diminish the removal of \ch{CO_2} from the atmosphere.
\item All described feedback mechanisms may increase surface air temperature and thus amplify local/regional temperature extremes.
\end{enumerate}

Guided by these hypotheses, the project team will specifically:
\begin{enumerate}
\itemsep0pt
\item Investigate the interaction between ozone and plant physiology and quantify on the one had the degree of plant damage to be expected through elevated ozone levels, and on the other the expected increase in ozone burden due to reduced dry deposition.
\item Quantify the effect of plan health on \gls{bvoc} emissions and subsequent effects on ozone air quality.
\item Quantify the effect of reduced stomatal conductance and thus photosynthesis on carbon sequestration and thus climate forcing.
\end{enumerate}

