\section{Added value to be expected from this collaboration}

The \gls{odina} project shows excellent fit with the research priorities of the host institution. \gls{boku}, Vienna is an international leader in environmental sciences with research priorities comprising the preservation and development of the environment and quality of life, management of natural resources and the environment and safeguarding food and health. 

Prof.~Rieder’s group at \gls{boku} focuses on chemistry-climate connections, particularly the effects of changes in atmospheric composition on climate and air quality across temporal and spatial scales. Within the framework of \gls{odina} Dr.~Falk and the host group would be the first to develop a fully coupled ozone module within the \gls{cesm} and apply it to study impacts and feedback mechanisms for plant health and air quality. This marks an important research effort given that the metric \emph{\gls{aot}} will soon be replaced by \emph{\gls{pod}} in ozone air quality legislation and assessment. \gls{aot} and \gls{mda8}~ozone have been notoriously high in Austria regardless of ozone precursor emission reductions. Given the diversity of research within the \gls{boku} community that focuses on plant physiology and plant health the model development and research activities within \gls{odina} will pave the road for manifold research endeavors ahead, and thereby create a lasting resource for \gls{boku} and the Austrian research landscape. Furthermore, the updated model version will increase the portfolio of contributions of the Rieder group to international activities within the framework of \gls{igac} activities such as phase~2 of the \gls{toar} and the \gls{ccmi}.

\section{Importance of the project for the academic and research reputation of the applicant and her career development}
The \gls{odina} project will allow the applicant to pursue her research interests and spearhead the inclusion of coupled ozone chemistry in one of the widest used community earth system models. Thereby she will be able to position herself as an emerging leader in the field of coupled modeling with focus on land-atmosphere interactions, particularly the climate-vegetation-air quality nexus. Interaction through model development and application with the \gls{cesm} core team and \gls{igac} community will expand and strengthen her professional network. The scientific and technical support of the host group and broader \gls{boku} community will provide an ideal environment for project implementation. Collaboration with the Rieder~group will allow to explore chemistry-climate connections and collaboration with other researchers at \gls{boku} will provide manifold opportunities to investigate effects of changing ozone abundances on crops and forests within \gls{odina}, and beyond. The mentoring and career development program accompanying this proposal (see \hyperref[appendix:career]{Annex~3}) will prepare Dr.~Falk for leadership roles in her academic career ahead.