\section{Added value to be expected from this collaboration}
The \gls{odina} project shows excellent fit with the research priorities of the host institution. The \gls{boku}, Vienna is an international leader in environmental sciences with research priorities comprising the preservation and development of the environment and quality of life, management of natural resources and the environment and safeguarding food and health. Prof. Rieder’s group at \gls{boku} focuses on chemistry-climate connections, particularly the effects of changes in atmospheric composition on climate and air quality across temporal and spatial scales. Within the framework of \gls{odina} Dr. Falk and the host group would be the first to develop a fully coupled ozone module within the \gls{cesm} and apply it to study impacts and feedback mechanisms for plant health and air quality. Given the diversity of research within the \gls{boku} community that focuses on plant physiology and plant health the model development and research activities within \gls{odina} will pave the road for manifold research endeavors ahead, and thereby create a lasting resource for \gls{boku} and the Austrian research landscape. Furthermore, the updated model version will increase the portfolio of contributions of the Rieder group to international activities within the framework of \gls{igac} activities such as \gls{toar}-2 and \gls{ccmi}-2.

\section{Importance of the project for the academic and research reputation of the applicant and her career development}
The \gls{odina} project will allow the applicant to pursue her research interests and spearhead the inclusion of coupled ozone chemistry in one of the widest used community earth system models. Thereby she will be able to position herself as an emerging leader in the field of coupled modeling with focus on land-atmosphere interactions, particularly the climate-vegetation-air quality nexus. Interaction through model development and application with the \gls{cesm} core team and \gls{igac} community will expand and strengthen her peer network. The scientific and technical support of the host group and broader \gls{boku} community will provide an ideal environment for project implementation. The mentoring and career development program accompanying this proposal will prepare Dr. Falk for an academic career ahead.

\section{Career Plan}
\label{sec:career}
In her research Dr. Falk strives to advance our understanding of causes and effects of changes in the chemical composition of the Earth’s atmosphere. To this end she applies state-of-the-art earth system models and contributes to their continuous development to improve the process level understanding and reduce uncertainties and biases in model projections. Dr. Falk aims to establish herself as an independent researcher with focus on the climate-air quality-health nexus and strives to achieve a permanent academic position either via a tenure-track faculty line or a senior scientist appointment. The proposed Lise-Meitner Fellowship is an important stepstone towards this goal.

The research framework of the fellowship project \gls{odina} allows Dr. Falk to follow her research interest and strengthen her scientific and technical portfolio to propel her career. Dr. Falk will spend her Lise-Meitner fellowship in Prof. Rieder’s group at the \gls{boku}, guaranteeing a stimulating research environment and ideal project support infrastructure. Prof. Rieder and his team focus on chemistry-climate connections, particularly ozone, and embedment in the \gls{boku} community will provide rich opportunities for interaction with research teams focusing on plant health and biosphere-atmosphere interaction. Including a module for coupled ozone chemistry between the atmosphere and land-surface components of \gls{cesm} Dr. Falk will provide an important extension to one of the widest used earth system models and provide rich opportunities for colleagues focusing on ambient air quality, ozone dry deposition and ozone plant damage and plant physiology. Given the broad impact of the model extension and opportunity to explore processes and feedbacks in the atmosphere-land system the \gls{odina} project will allow Dr. Falk to establish herself as an emerging leader in the field of land-atmosphere interaction.

The model development activities of \gls{odina} will link Dr. Falk with the core development team of the \gls{cesm} model at the \gls{ncar} in Boulder, CO, USA. The connection to \gls{ncar}'s Atmospheric Chemistry Observations \& Modeling Lab and Climate \& Global Dynamics Lab will be fostered through a 3-month research stay in the first year of the project. The \gls{odina} project will embed Dr. Falk also in the \gls{toar}-2 and \gls{ccmi}-2 activities of the \gls{igac} Project and thereby substantially expand her professional network.

Throughout the project the interaction between the Dr. Falk and Prof. Rieder will focus on all aspects related to the mentees career development and the overall progress of the research project. To this end they will meet for discussions on a weekly basis. These meetings will focus on the one hand on project status, opportunities or challenges emerging and scientific findings, on the other on recently published research highlights in the field, emerging funding and employment opportunities, as well as upcoming conferences or trainings. Prof. Rieder will provide dedicated support to Dr. Falk throughout the project. He will provide guidance and counseling in all scientific and professional aspects, introduce Dr. Falk to his network of peers and promote her as guest speaker for institute seminars and conferences. He will further provide guidance in the leverage of research results for high-profile publication formats and the effective dissemination of research outcomes via traditional media and online platforms. Besides Prof. Rieder, Dr. Falk will obtain mentoring by Dr. Mayer (\gls{boku}-Met) and other peers in the \gls{eswn}.

As an affiliate of \gls{boku} Dr. Falk will have access to a wide range of training and career development programs offered by the university. These include among others leadership training and didactic training as well as a variety of courses tailored to strengthen skills in project acquisition and project management and the preparation of application packages for tenure-track or staff scientist positions or the preparation of a habilitation thesis. Dr. Falk will be further assisted through \gls{boku}’s programs within the framework of the Postdoc Coaching Group and individual career coaching sessions. During her fellowship at \gls{boku} Dr. Falk will also gain experience in the mentoring and co-advising of undergraduate and graduate students within Prof. Rieder’s group and be supported in dissemination and outreach activities by the university media and public relations office.

To ensure continued funding after the end of the Liese-Meitner funding period Prof. Rieder and the \gls{boku} office for research support, innovation \& technology transfer will support Dr. Falk in the development of applications to national and European Union research programs. Furthermore to ensure further collaboration Prof. Rieder and his team will and include Dr. Falk in their own applications submitted to these funding schemes.
