The applicant, \textbf{Stefanie Falk}, is a physicist by training. Her postdoctoral stations included the \gls{kit} and the \gls{uio}. Her curiosity about the complex and important role of ozone in the atmosphere led her from studying its chemical depletion in the stratosphere and polar boundary layer to its formation and removal processes by the land biosphere. Interdisciplinary work has been a cornerstone in her work to improve the understanding and push the development of current numerical models of the Earth system forward. In particular, she has been working on \gls{ccm} using the \gls{emac} model. Her work was related to future trends of stratospheric ozone, biogenic brominated \gls{vsls}, and the influence of sulfur aerosols. During her engagement at \gls{kit}, she also implemented a bromine release mechanism from snow and studied ozone depletion in the Arctic boundary layer. Identifying dry deposition as a key issue for both future air quality as well as climate, she committed to studies on the effects of ozone on vegetation in her latest work at \gls{uio}. From ozone formation and dry deposition on global and regional scales, her focus shifted later to process-based impact modeling at the leaf level and improvements of subarctic biomes. Her work resulted in scientific publications and conference contributions both nationally and internationally. She gladly participates in public outreach, at \gls{uio} through public lectures and popular science blogs in the Norwegian language. She is a member of the \gls{dpg} and the professional body of the environmental section in the \gls{dpg}, \gls{yess} community, and the \gls{eswn}.\\

\textbf{Harald Rieder} is full professor and head of the institute of meteorology and climatology at the University of Natural Resources and Life Sciences, Vienna. He is an expert in chemistry-climate connections. His research is centered on the climate-air quality-health nexus and combines numerical models with observational data sets from both ground-based networks and remote sensing platforms. He has led and extensively contributed to studies unravelling the effect of changes in climate, ambient meteorology, and anthropogenic emissions on surface pollution as well as the effects of atmospheric composition changes for climate at the surface and at upper atmospheric levels. Prof. Rieder serves among others as board member of the Austrian Centre for Climate Change and Member of the \gls{isac} Atmosphere Working Group.
