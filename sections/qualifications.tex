\glsresetall
\textbf{The applicant, Stefanie Falk} is an expert in land-atmosphere interactions. She is a physicist by training and her postdoctoral stations included the \gls{kit} and the \gls{uio}. Her curiosity about the complex and important role of ozone in the atmosphere led her from studying its chemical depletion in the stratosphere and polar boundary layer to its formation and removal processes by the land biosphere. Interdisciplinary collaboration has been a cornerstone in her work to improve our process level understanding and advance earth system models. In particular, she has been working on \gls{ccm} using the \gls{emac} model. Her work was related to future trends of stratospheric ozone, biogenic brominated \gls{vsls}, and the influence of sulfur aerosols. During her time at \gls{kit}, she also implemented a bromine release mechanism from snow and studied ozone depletion in the Arctic boundary layer. Identifying dry deposition as a key issue for both future air quality and climate, she most recently studied the effects of ozone on vegetation at \gls{uio}. From ozone formation and dry deposition on global and regional scales, her focus shifted later to process-based impact modeling at the leaf level and improvements of subarctic biomes. Her work resulted in scientific publications and conference contributions both nationally and internationally. She gladly participates in public outreach, most recently at \gls{uio} through public lectures and popular science blogs. She is a member of the \gls{egu}, the \gls{dpg} and the professional body of the environmental section in the \gls{dpg}, \gls{yess} community, and the \gls{eswn}.\\

\textbf{The co-applicant, Harald Rieder} is full professor and head of the institute of meteorology and climatology at the University of Natural Resources and Life Sciences, Vienna. He is an expert in chemistry-climate connections. His research is centered on the climate-air quality-health nexus and combines numerical models with observational data sets from both ground-based networks and remote sensing platforms. Dr. Rieder has led and extensively contributed to studies unravelling the effect of changes in climate, ambient meteorology, and anthropogenic emissions on surface pollution as well as the effects of atmospheric composition changes for climate at the surface and at upper atmospheric levels. He serves among others as board member of the \gls{ccca} and Member of the \gls{iasc} Atmosphere Working Group. His dedication to provide support and mentoring to the next generation of scientific leaders is reflected in his service at \gls{boku}. Dr. Rieder is a member of the university qualification board (accompanying scientist with tenure track and development agreements), and serves on several curricula commissions and the board of a \gls{boku} doctoral school. 
